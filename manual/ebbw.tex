\documentclass[a4paper]{article}

\usepackage[margin=35mm]{geometry}

\usepackage[utf8]{inputenc}
\usepackage[english]{babel}

\usepackage[math]{iwona}
\usepackage[T1]{fontenc}
\usepackage{inconsolata}

\usepackage{mathtools, xcolor}

\def\D{\mathrm d}
\def\E{\mathrm e}
\def\I{\mathrm i}

\def\sub#1{\sb{\mathrm{#1}}}
\def\limit#1{\sp{\mathclap{#1}}}

\let\vec\boldsymbol
\let\tilde\widetilde
\let\epsilon\varepsilon
\let\Theta\varTheta

\def\headline#1{\section*{\normalsize\normalfont%
   \rlap{\rule[0.5ex]{\textwidth}{0.4pt}}\qquad\colorbox{white}{#1}}}

\pagestyle{empty}

\begin{document}
   \begin{center}
      \LARGE \texttt{ebbw} \par \bigskip
      \large Solve Eliashberg equations for finite bandwidths
   \end{center}

   \headline{Eliashberg theory}

   The linearized local Eliashberg equations for the density of states
   $N(\epsilon) = N(0) \, \Theta(E - |\epsilon|)$ are
   %
   \begin{align*}
      \begin{aligned}
         \tilde \omega_n &= \omega_n + 2 T \sum_m \limit{|\omega_m| < \omega_N}
         \arctan \Big[ \frac E {\tilde \omega_m} \Big] \, \lambda_{n - m},
         \\
         \phi_n &= 2 T \sum_m \limit{|\omega_m| < \omega_N}
         \arctan \Big[ \frac E {\tilde \omega_m} \Big] \,
         \frac{\phi_m}{\tilde \omega_m} \, (\lambda_{n - m} - \mu^*_N),
      \end{aligned}
      %
      \qquad \text{where} \qquad
      %
      \begin{aligned}
         \frac \lambda {\lambda_n} &=
         1 + \Big[ \frac{\nu_n}{\omega \sub E} \Big]^2,
         \\
         \frac 1 {\mu^*_N} &=
         \frac 1 \mu + \frac 1 \pi \int_{-E}^E \frac {\D \epsilon} \epsilon
         \arctan \Big[ \frac \epsilon {\omega_N} \Big].
      \end{aligned}
   \end{align*}
   %
   $\omega_n = (2 n + 1) \pi T$ and $\nu_n = 2 n \pi T$ are fermionic and
   bosonic Matsubara frequencies, $\tilde \omega_n$ are renormalized fermionic
   frequencies and $\phi_n$ is proportional to the infinitesimal superconducting
   order parameter at the critical point $(T, \lambda, \mu, \omega \sub E, E)$.
   Further quantities are explained below.

   \headline{Parameters}

   \begin{center}
      \begin{tabular}{r c c l}
         \verb|T| & $T$ & K & temperature \\
         \verb|l| & $\lambda$ & 1 & electron-phonon coupling \\
         \verb|u| & $\mu$ & 1 & Coulomb repulsion \\
         \verb|w| & $\omega \sub E$ & eV & Einstein phonon frequency \\
         \verb|E| & $E$ & eV & half the electronic bandwidth \\
         [2mm]
         \verb|cutoff| & $\omega_N$ & $\omega_E$ & cutoff frequency \\
         \verb|rescale| & & & if false, use $\mu$ in place of $\mu^*_N$ \\
         \verb|epsilon| & & any & self-consistency threshold for final result \\
         \verb|terms| & & 1 & number of terms in series expansion of
         $1 / \mu^*_N - 1 / \mu$
      \end{tabular}
   \end{center}

   \headline{Functions}

   \begin{itemize}
      \item[def]
         \verb|critical(variable='T', epsilon=1e-3,|
         \verb| T=10, l=1, u=0.1, w=0.02, E=1, ...)|

      solves the Eliashberg equations by varying the parameter indicated by
      \verb|variable|, which may be $T$, $\lambda$, $\mu$, $\omega \sub E$ or
      $E$, leaving the others fixed. Any given value for this parameter is used
      as initial guess. The critical value is returned.

      \item[def] \verb|Tc(l, u, w, E, A=0.94, B=1.11, C=0.74, ...)|

      returns the critical temperature according to McMillan's formula,
      %
      \begin{equation*}
         T \sub c = \frac {\omega \sub E} A \exp
         \Big[ \frac{-B (1 + \lambda)}{\lambda - C \lambda \mu^* - \mu^*} \Big]
         \quad \text{with} \quad
         \frac 1 {\mu^*} =
         \frac 1 \mu + \log \Big[ \frac E {\omega \sub E} \Big],
      \end{equation*}
      %
      where default values for $A$, $B$ and $C$ have been adjusted to an
      Einstein phonon spectrum.

      \item[def] \verb|Eliashberg(T, l, u, w, E, cutoff=15, rescale=True, ...)|

      returns the maximum eigenvalue of the kernel of the equation for $\phi_n$,
      which is less than, equal to or greater than unity in the normal, critical
      or superconducting state, respectively.

      \item[def] \verb|residue(x, terms=10)|

      returns $1 / \mu^*_N - 1 / \mu$ as a function of $E / \omega_N$,
      approximated via a series expansion.

      \item[def] \verb|eigenvalue(matrix)|

      returns the maximum eigenvalue of the given matrix.
   \end{itemize}
\end{document}
